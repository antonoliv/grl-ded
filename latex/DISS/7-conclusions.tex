\chapter{Conclusions} \label{chap:conclusion}


In this report, we document the study performed around the improvement of \ac{GRL} techniques to solve the \ac{DED} problem. After a thorough study of the relevant literature and a complete overview of relevant knowledge, the results showed that \ac{GRL} approaches demonstrated confident results facing the modern pure-\ac{DRL} algorithms. Beyond that, the literature was particularly optimisitic of the scalability and adaptability of these techniques in larger grids with real-time topological changes. \par
It's crucial to highlight the complexity of the \ac{DED} problem, which was not fully explored in the context of this work. The literature showed a diverse set of constraints, datasets, action and observation spaces tied with this problem. However, the conducted review failed to find relevant works with appropriately and openly distributed sources, which severely difficult the replication of results. \par
The initial experiments and the calibration process unveiled several concerns mainly related to the action space complexity and the \ac{GNN} tuning. The proposed GCN-SAC and GAT-SAC implementations, even with the thorough tuning process they were put through failed to surpass the sample efficiency and stability of the state-of-the-art \ac{SAC} algorithm. The former was superior in training convergence and overall results. Nonetheless, our results demonstrate a slight advantage of the GCN-SAC model on scalability, which showed consistent results on processing time in a larger scenario and performance closer to the state-of-the-art model. \par
Beyond that, a test framework resulted from this work, enabling anyone to replicate the performance of the models implemented. The framework allows any combination of \ac{DRL} and \ac{GNN} algorithm (implemented by \textit{stable-baselines3} or \textit{torch\_geometric}) on several available power grid environments, as well as, performing training, test and visual analysis on the models. This enables similar studies on other algorithms and more specialized tuning processes to be conducted, further advancing research and knowledge around the field of \ac{GRL}.
Finally, in a last reflection on the results achieved, the author still believes in the promising results and solid foundation established in the literature review for the novel \ac{GRL} techniques. Although in this study this was not verified, there's still a lot of practical insights in model implementation to be gathered in order to improve the model's computational performance and learning efficiency. Additionally, the proposed implementations could benefit from a more extensive tuning process and subsequent analysis in order to substantiate the potential superiority of these models.


\section{Main Contributions}

In this section, we list the expected contributions derived from this work on a Scientific, Technological and Application levels:
\begin{description}
	\item[Scientific] A systematic and comparative study of different \ac{GRL} approaches. This fills the research gap for systematic studies comparing different proposed techniques. Furthermore, this work will bring a clearer insight regarding the best practices on implementing \ac{GRL} models
	\item[Technological] A model resulting from this study with concrete improvements over the current \ac{GRL} techniques proposed so far and tackling their limitations. Beyond this, 
	\item[Application] An efficient solution for the \ac{DED} problem with \ac{GRL} algorithms, compromising significant contribution to the research on \ac{DED} systems in complex scenarios
\end{description}

\section{Future Work}

While this work extensively explored potential improvements in \ac{GRL} techniques and their applications to the \ac{DED} problem, the scope of the issue is vast, and several areas for improvement were left unaddressed due to time and planning constraints. The main future concerns are aligned with improving the performance of the model and further investigating other potential enhancements to the \ac{GRL} techniques. \par

Firstly, it would be valuable to explore additional \ac{DRL} and \ac{GNN} implementations to compare with the results obtained from the studied models. For example, alternative combinations of \ac{GNN} models like GraphSAGE and \ac{DRL} algorithms such as PPO and DDPG might yield better performance than those tested in this study. Additionally, the author is keen to compare the studied single-agent \ac{GRL} technique with multi-agent systems in the \ac{DED} problem. Given their distinct perspectives on the issue, such a comparison could provide valuable insights into the effectiveness of each approach. \par

Given the complexity and diversity of modern power systems, it would also be worthwhile to consider additional elements, such as \acf{ESS} storage, which were excluded to avoid overly complicating the environmental dynamics. Moreover, investigating how voltage control could enhance the agent's reliability and, consequently, the survivability of models appears to be a relevant area for further study. \par

Finally and most importantly, further experimentation and calibration, especially with larger scenarios, might be necessary to understand the failures of this work's approach on \ac{GRL} techniques. There is still promising evidence from the literature of their performance on graph-based scenarios, but the study and analysis that was conducted failed to reach that conclusion. 

