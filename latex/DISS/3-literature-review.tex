% !TeX spellcheck = en_GB
\chapter{Literature Review} \label{chap:review}

In this chapter, the literature review regarding \ac{GRL} approaches and \ac{DED} systems is documented. The structure of this unit is subdivided into sections that contain the several aspects of the review process, with section \ref{sec:review-method} exposing the methodology for this literature analysis. \par
Sections \ref{sec:review-grl} and \ref{sec:review-ded} document the actual findings of the review process, with the first being illustrating \ac{GRL} approaches in general and the second focusing on \ac{RL} techniques for solving the \acf{DED} problem specifically. Section \ref{sec:review-grl} is also subdivided into the two most prominent and found approaches, which are \ac{GCN}-bases techniques, in section \ref{sec:review-gcn}, and attention-based techniques, in section \ref{sec:review-gat}. \par
Finally, in the last section \ref{sec:review-conclusion}, the main considerations taken from this literature review are uncovered and reflected upon.

\section{Research Methodology} \label{sec:review-method}

This literature review focuses on analysing the existent works around the main objective of this dissertation, which is improving \ac{GRL} techniques in the context of this work's application domain, smart grid services. In this manner, the main research questions are presented as:

\begin{itemize}
	\item RQ1 - \textit{How can \ac{RL} algorithms be adapted to effectively solve sequential decision-making problems on graph-based environments?} 
	\begin{itemize}
		\item RQ1.1 - \textit{What are the existent \ac{GRL} approaches?}
		\item RQ1.2 - \textit{What are their limitations? }
	\end{itemize}
	\item RQ2 - \textit{How can \ac{GRL} algorithms improve smart grid services?}
	\begin{itemize}
		\item RQ2.1 - \textit{How can \ac{GRL} algorithms improve \acf{DED} in power distribution grids?}
	\end{itemize}
\end{itemize}

These interrogations aggregate the relevant topics we aim to address in this review. The main requirement on the analysed literature was to only consider research from the past five years, except seminal works. We also exclusively considered literature published in scientific conferences and top-tier journals. \par
In this manner, the initial exploratory research was conducted, primarily using \textit{Scopus} and \textit{Web of Science} for searching the published literature and alternatively using \textit{Google Scholar} for finding cross-references when necessary. 
The research questions were translated into fundamental search queries where the research process was based.


\begin{itemize}
	\item \texttt{"Graph Reinforcement Learning" OR "Reinforcement Learning on Graphs"}
	\item \texttt{"Reinforcement Learning" AND "Power" AND "Dispatch"}
	\item \texttt{"Graph Reinforcement Learning" OR "Reinforcement Learning on Graphs" AND "Power" AND "Dispatch"}
\end{itemize}

 The gathered literature was screened and the relevant works were thoroughly reviewed, the following sections present the main findings.


\begin{comment}
	* Add literature histogram chart
\end{comment}



\section{Graph Reinforcement Learning Approaches} \label{sec:review-grl}

\ac{GRL} or Reinforcement Learning on Graphs is a relatively new area in the broader field of machine learning. \ac{GRL} techniques have shown significant progress in solving problems with underlying graph-based representations such as power grid management \cite{liNovelGraphReinforcement2022, chenGraphRepresentationLearningbased2023}, smart transport \cite{xingBilevelGraphReinforcement2023, almasanDeepReinforcementLearning2022} or task offloading \cite{gaoFastAdaptiveTask2023, liGraphReinforcementLearningbased2022}. In this work, the main focus lies on studying the development of \ac{GRL} techniques and subsequent application to smart grid services such as dynamic economic energy dispatch systems \cite{chenScalableGraphReinforcement2023, xingRealtimeOptimalScheduling2023}, residential electricity behaviour identification and energy management \cite{chenGraphRepresentationLearningbased2023}, or Volt-VAR regulation \cite{huMultiagentGraphReinforcement2024}.  \par
Research on this topic has significantly increased in the last few years with the improvements of \ac{DRL} techniques and the developments in \acp{GNN} in the mid-2010s \cite{kipfSemiSupervisedClassificationGraph2017, velickovicGraphAttentionNetworks2018, liGatedGraphSequence2016, gaoGraphUNets2019}. \acp{GNN} became the state-of-the-art for solving numerous data mining tasks involving graph-structured data, excelling at classification, link prediction and representation learning \cite{xuHowPowerfulAre2019, nieReinforcementLearningGraphs2023}. This advancement brought more sophisticated \ac{RL} applications on graphs and the surge of a new field studying how to combine the improvements of graph mining and reinforcement learning techniques \cite{vesselinovaLearningCombinatorialOptimization2020, nieReinforcementLearningGraphs2023}. \par

\subsection{Plain GCN-Based GRL} \label{sec:review-gcn}

A common approach in Graph Reinforcement Learning model implementation is the use of graph convolutions with the \acp{GCN} architecture for leveraging graph-based structures to extract and aggregate the essential features of data in hand and improve the performance of \ac{RL} agents in those environments. The techniques listed in this subsection constitute approaches that integrate a \ac{GCN} with \ac{RL} algorithms. The gathered literature can be observed in table \ref{tab:gcn-lit}. \par
\cite{liNovelGraphReinforcement2022} implements a \ac{GRL} system to improve the decision quality of economic dispatch under high penetration of distributed energy generations. To accomplish this, a \ac{SAC} system is employed with the main objective of finding the optimal action policy for minimizing generation cost with the appropriate reliability concerns. This problem is represented by an undirected graph with nodes describing the power grid elements with their respective attributes and edges describing the underlying energy connections between those units. To extract the structural features of the graph, this work implements a full connected layer to perform feature transformation with a two-layer \ac{GCN} followed by three full connected layers for the non-linear mapping of state-to-action policy in both actor and critic modules. \cite{chenScalableGraphReinforcement2023} develops a similar approach, with both concluding that it significantly reduces learning time for achieving better training results in comparison to plain \ac{SAC} and showing significant improvement on economy and flexibility of the system on more complex and sparse state graphs. The use of \acp{GCN} enables the system to adapt to changes in the state space by leveraging the network's generalization ability.\par
In \cite{xingGraphReinforcementLearningBased2023} a three-layer \ac{GCN} is used to extract node feature and graph topology information and is integrated into a Rainbow-based \cite{hesselRainbowCombiningImprovements2018} \ac{DRL} algorithm for electric vehicle charging guidance. In this article, the testing results show promising performance in reducing operation costs for electric vehicle users, portraying a model with good generalization ability in untrained scenarios. This work and  \cite{chenScalableGraphReinforcement2023} further tested their proposed model's performance when inducing topology changes, yield promising results in the adaptability of the two \ac{GRL} algorithms in scenarios where the environment suffers dynamic changes.\par
Another interesting implementation of this approach is \cite{chenAutonomousExplorationUncertainty2020}, which studies and compares different solutions for optimizing autonomous exploration under uncertain environments. It analyses combinations of a single agent Deep Q-Network (DQN) and Advantageous Actor-Critic (A2C) with Graph Convolutional Networks, Gated Graph Recurrent Networks and Graph U-Nets. The algorithms are executed in test cases of various sizes and the paper reports that the GCN-DQN was the model that achieved the highest reward during policy training, followed by the GGNN-A2C model, although in the end, it concludes that the second showed improved scalability in relation to the first model. Other similar approaches include \cite{chenGraphRepresentationLearningbased2023} for residential electricity behaviour identification and energy management with a behaviour correlation graph and \cite{xuSimulationConstraintGraphReinforcement2020} for line flow control in a power grid simulation. \par
The reviewed literature fails to consider methods for analysing the scalability of proposed models in relatively larger scenarios, except \cite{chenAutonomousExplorationUncertainty2020}, with some proposing this as relevant future work \cite{chenScalableGraphReinforcement2023, xingGraphReinforcementLearningBased2023}. Beyond that, only \cite{xingGraphReinforcementLearningBased2023}, \cite{chenGraphRepresentationLearningbased2023} \cite{chenScalableGraphReinforcement2023} defined methods for evaluating the performance of the algorithms under topology variations. This shows a gap for more complete studies in plain \ac{GCN}-based approaches that also focus on studying the scalability of \ac{GRL} to large scenarios and the adaptability of the models to dynamic topology variations. In most approaches, namely in \cite{xingGraphReinforcementLearningBased2023, chenScalableGraphReinforcement2023, liNovelGraphReinforcement2022, xuSimulationConstraintGraphReinforcement2020}, the presented results portray \ac{GRL} techniques as significantly more effective than plain \ac{DRL} models without a \ac{GCN} already suggesting that these techniques are a potential solution for solving sequential decision-making problems with graph-based environments.

\begin{table}[H] 
	\centering
	\rowcolors{2}{gray!15}{white}
	\caption{\acs{GCN}-based \acs{GRL} literature.}
	\begin{tabular}{P{2cm}P{2cm}P{2cm}p{5cm}}
		\toprule
		\textbf{Reference} & \textbf{DRL} & \textbf{GNN}& \textbf{Application Domain} \\
		\midrule
		\cite{yuanXGNNModelLevelExplanations2020} & MDP & GCN & Interpret GNNs at model-level \\
		\cite{wangGCNRLCircuitDesigner2020} & DDPG & GCN & Automatic transistor sizing \\
		\cite{yanAutomaticVirtualNetwork2020} & A3C & GCN & Automatic Virtual Network Embeddings \\
		\cite{tangDependentTaskOffloading2020} & Deep Q-Learning & GCN & Task Offloading in Edge Computing \\
		\cite{xingGraphReinforcementLearningBased2023} & Rainbow  &  GCN & Electrical Vehicle Charging Guidance \\
		\cite{liNovelGraphReinforcement2022}, \cite{chenScalableGraphReinforcement2023} & SAC & GCN & Dynamic economic energy dispatch \\
		\cite{lengGraphConvolutionalNetworkbased2021} & SAC & GCN & Multi-access Edge Computing \\
		\cite{xuSimulationConstraintGraphReinforcement2020} & DQN & GCN & Line flow control \\
		\cite{chenAutonomousExplorationUncertainty2020} & DQN & GCN  & Autonomous Exploration under uncertainty \\ 
		\cite{chenGraphRepresentationLearningbased2023} & DQN & GCN & Residential electricity behavior identification and energy management \\ 
		\bottomrule
	\end{tabular}
	\label{tab:gcn-lit}
\end{table}




\subsection{Attention-based GRL} \label{sec:review-gat}

Another effective approach in extracting relevant topology and graph features relies on using attention mechanisms to weigh different nodes' contributions dynamically. While this encompasses techniques that use the \ac{GAT} architecture, which is a \ac{GNN} design with the attention mechanism at its core, various scholars propose \ac{GCN} approaches integrated with attention mechanisms such as \cite{zhaoLearningSequentialDistribution2022} and \cite{fanAttentionBasedMultiAgentGraph2023}. In the top part of table \ref{tab:gat-lit} the single-agent reviewed attention-based approaches can be observed, while at the bottom some relevant multi-agent approaches are listed.  \par
\cite{xingRealtimeOptimalScheduling2023} proposes a \ac{DDPG}-based algorithm improved with a \ac{GAT} block with three graph Attention Layers for extracting and learning the topology information for achieve real-time optimal scheduling for \acp{ADN}. This paper compares the obtained test results against a \ac{GCN}-\ac{DDPG} model and shows increased performance over the \ac{GCN} method in reducing cost and power loss. Beyond this, the work demonstrates that the \ac{GAT}'s attention mechanism enables the algorithm to focus on more important nodes and improve the signal-to-noise ratio compared to its \ac{GCN} counterpart. \cite{chenPhysicalassistedMultiagentGraph2023} and propose a multi-agent approach to the same domain but more focused on voltage regulation with a multi-agent \ac{SAC} instead of a single-agent \ac{DDPG} algorithm. \par
In \cite{xingBilevelGraphReinforcement2023}, another model for the electric vehicle charging guidance is proposed, consisting of a bi-level approach of a Rainbow-based algorithm with a \ac{GAT} block. The upper level focuses on the decision-making process regarding charging, while the lower level handles routing. The proposed model proved to be more effective than a shortest distance path-based \cite{xingModellingDrivingCharging2021} and a \ac{DRL}-based \cite{qianDeepReinforcementLearning2020} approach. \cite{xuRealtimeFastCharging2022} develops a similar approach with a Double-prioritized DQN for the same application domain.
In \cite{zhaoLearningSequentialDistribution2022} and \cite{fanAttentionBasedMultiAgentGraph2023}, the sequential distribution system restoration problem is addressed with a multi-agent \ac{RL} algorithm equipped with a \ac{GCN} with an attention mechanism. In the first case, multi-head attention is used as the convolution kernel for the \ac{GCN} with a \ac{DQN} algorithm. In the second, self-attention is used for improving the centralized training of the used multi-agent actor-critic algorithm, more concretely, by embedding it in the critic networks. At the same time, the \ac{GCN} is integrated into the actor networks for extracting the graph features. Both solutions proved more efficient than traditional \ac{RL} techniques, with the first highlighting its solution generalization ability and the second showing increased scalability facing the non-GRL techniques. 
In general, the literature regarding attention-based approaches is a lot sparser and scarcer than \ac{GCN}-based approaches. Only three relevant single-agent works were found \cite{xingRealtimeOptimalScheduling2023, xingBilevelGraphReinforcement2023, xuRealtimeFastCharging2022} which might either suggest that scholars have a slight preference for multi-agent systems when implementing \ac{GRL} with attention mechanisms or that these approaches in single-agent systems still require further research. Nevertheless, some of the works showed models with good adaptability to topology variations \cite{xingRealtimeOptimalScheduling2023, baiAdaptiveActivePower2023b, chenPhysicalassistedMultiagentGraph2023} and scalability to large scenarios \cite{zhaoLearningSequentialDistribution2022, xingRealtimeOptimalScheduling2023, baiAdaptiveActivePower2023b, chenPhysicalassistedMultiagentGraph2023}. Notably, some works suggest directly embedding the \ac{GNN} in th \ac{RL} framework to boost the methodology computation performance and robustness  \cite{xingRealtimeOptimalScheduling2023, xingRealtimeOptimalScheduling2023}.


\begin{table}[H] 
	\centering
	\rowcolors{2}{gray!15}{white}
	\caption{Attention-based \acs{GRL} literature.}
	\begin{tabular}{P{2cm}P{2cm}P{2cm}p{5cm}}
		\toprule
		\textbf{Reference} & \textbf{DRL} & \textbf{GNN} & \textbf{Application Domain} \\
		\midrule
		\cite{xingRealtimeOptimalScheduling2023} & DDPG & GAT & Optimal Scheduling for ADNs  \\ 
		\cite{xingBilevelGraphReinforcement2023} & Rainbow & GAT  & Electric Vehicle Charging Guidance \\
		\cite{xuRealtimeFastCharging2022} & DQN & GAT & Electric Vehicle Charging Guidance \\
		\cite{zhaoLearningSequentialDistribution2022} & DQN & GCN & Multi-agent Sequential Distribution System Restoration \\ 
		\cite{baiAdaptiveActivePower2023b} & DQN & GCN & Active Power Rolling Dispatch \\ 
		\cite{fanAttentionBasedMultiAgentGraph2023} & AC & GCN & Multi-agent Service Restoration \\ 
		\cite{chenPhysicalassistedMultiagentGraph2023}  & SAC & GAT  & Multi-agent Voltage Regulation \\
		\bottomrule
	\end{tabular}
	\label{tab:gat-lit}
\end{table}


\subsection{Other Approaches} \label{sec:review-other-gnn}

This subsection includes other relevant and promising \ac{GRL} approaches that combine of other \acp{GNN} architectures with \ac{RL} algorithms . In \cite{peiEmergencyControlStrategy2023}, a GraphSAGE network \cite{hamiltonInductiveRepresentationLearning2018} is used with a Deep Dueling Double Q-Network (D3QN) for emergency control of Undervoltage load shedding for power systems with various topologies. The author presents promising results for the GraphSAGE-D3QN model compared to a GCN-D3QN, achieving higher cumulative reward and faster voltage recovery speed, although it required longer decision times. The proposed model performed excellently in the application domain and successfully generalized the learned knowledge to new topology variation scenarios. Another approach that showed good performance with the GraphSAGE architecture was \cite{zhaoGraphbasedDeepReinforcement2022} in the context of the dynamic economic dispatch problem. \par
\cite{zhangLearningDispatchJob2020} focused on solving the Job shop scheduling problem through a priority dispatching rule with a \ac{GIN} \cite{xuHowPowerfulAre2019} and an actor-critic \ac{PPO} algorithm where the \ac{GIN} is shared between actor and critic networks. The method showed superior performance against other traditional manually designed priority dispatching rule baselines, outperforming them by a large margin.

\begin{table}[H] 
	\centering
	\rowcolors{2}{gray!15}{white}
	\caption{Other \acs{GRL} approaches in literature.}
	\begin{tabular}{P{2cm}P{2cm}P{2cm}p{5cm}}
		\toprule
		\textbf{Reference} & \textbf{DRL Algorithm} & \textbf{GNN Algorithm} & \textbf{Application Domain} \\
		\midrule
		\cite{peiEmergencyControlStrategy2023} & DQN & GraphSAGE & Undervoltage Load Shedding  \\ 
		\cite{zhaoGraphbasedDeepReinforcement2022} & PPO & GraphSAGE  & Dynamic Economic Dispatch \\ 
		\cite{zhangLearningDispatchJob2020} & PPO & GIN & Dispatch for Job Shop Scheduling \\ 
		\bottomrule
	\end{tabular}
	\label{tab:other-lit}
\end{table}

\section{Dynamic Economic Dispatch Systems} \label{sec:review-ded}

\ac{RL} algorithms are already regarded as a well established potential solution for solving economic dispatch problems \cite{pereraApplicationsReinforcementLearning2021}. In table \ref{tab:ded-lit} the relevant \ac{RL} approaches to the problem, including \ac{GRL} techniques, can be observed. In a general level, the reviewed literature regarding \ac{DED} solutions take into account a wide range of considerations and constraints. The recent works seem to favour a focus around \acp{ESS} management and \ac{RES} curtailment, given the current global energetic issues and the relevance of these technologies for solving them. \par
Some systems \cite{chenScalableGraphReinforcement2023, hanAutonomousControlTechnology2023, zhaoGraphbasedDeepReinforcement2022} take further steps in ensuring grid stability by also considering voltage deviations while a notable paper considers battery degradation when calculating the cost of dispatch \cite{liuDistributedEconomicDispatch2018}. In general, \ac{GRL} algorithms show superior performance in relation with plain \ac{DRL} approaches \cite{chenScalableGraphReinforcement2023, liNovelGraphReinforcement2022, zhaoGraphbasedDeepReinforcement2022} as also concluded in section \ref{sec:review-gcn}, with studies successfully ensuring adaptability of the \ac{GRL} models to variations in the scenario's topology \cite{chenScalableGraphReinforcement2023, zhaoGraphbasedDeepReinforcement2022}. However, only \cite{baiAdaptiveActivePower2023b} conducts tests in test cases of different sizes, which shows a gap for studies addressing the scalability limitations of the developed models.

		
\begin{table}[H]
		\centering
		\rowcolors{2}{gray!15}{white}
		\caption{Dynamic economic dispatch \ac{RL} literature.}
		\begin{tabular}{P{2cm}P{4cm}p{6cm}}
		\toprule
		\textbf{Reference} & \textbf{Approach} & \textbf{Application} \\
		\midrule
		\cite{yangDynamicEnergyDispatch2021} & DDPG with Prioritized Experience Replay Mechanism and L2 Regularization& Integrated Energy Systems, Utility (Selling, Purchasing and Gas) and \acp{ESS} \\
		\cite{leiDynamicEnergyDispatch2021} & DDPG & Thermal Power, Renewable Energy Sources and \acp{ESS} \\
		\cite{hanAutonomousControlTechnology2023} & SAC with Imitation learning & Thermal Power, Renewable Energy Sources and Voltage deviation\\
		\cite{chenScalableGraphReinforcement2023} & GCN-SAC with Replay Buffer & Thermal Power, Renewable Energy Sources, \acp{ESS} and Voltage Deviation \\
		\cite{liNovelGraphReinforcement2022} & GCN-SAC & Utility (Time-of-Use), Thermal Power, Renewable Energy Sources and \acp{ESS} \\
		\cite{zhaoGraphbasedDeepReinforcement2022} & GraphSAGE-PPO & Thermal Power, Renewable Energy Sources, \acp{ESS} and Voltage Deviation \\ 
		\cite{liuDistributedEconomicDispatch2018} & Multi-Agent \ac{RL} with Function Approximation and Diffusion Mechanism & Utility (Time-of-Use), Thermal Power (Diesel) and \acp{ESS} (Considering degradation) \\ 
		\cite{baiAdaptiveActivePower2023b} & Multi-Agent GAT-DQN & Thermal Power, Renewable Energy Sources and \acp{ESS} \\ 
		\bottomrule
	\end{tabular}
	\label{tab:ded-lit}
\end{table}

The main considerations regarding action space and observation space found in literature can be observed in tables \ref{tab:ded-action} and \ref{tab:ded-obs}, respectively. Regarding the first, literature doesn't deviate much from including dispatch actions on non-renewable generators under the form of increments or decrements of the power output. Those intervals are restricted by maximum ramp limits and maximum and minimum output limits \cite{liNovelGraphReinforcement2022, chenScalableGraphReinforcement2023}. In some works \cite{zhaoGraphbasedDeepReinforcement2022}, the problem is simplified by using discretization to reduce the complex dimensionality of \ac{DED}. These use output changes of fixed intervals, considering only actions that constitute a multiple of that interval.
Furthermore, other already mentioned features can be included action space , such as \ac{ESS} power output setpoint \cite{zhaoGraphbasedDeepReinforcement2022, liNovelGraphReinforcement2022, chenScalableGraphReinforcement2023, liuDistributedEconomicDispatch2018, yangDynamicEnergyDispatch2021, xingRealtimeOptimalScheduling2023} \acf{RES} curtailment \cite{zhaoGraphbasedDeepReinforcement2022, liNovelGraphReinforcement2022, chenScalableGraphReinforcement2023, xingRealtimeOptimalScheduling2023}, and voltage control \cite{hanAutonomousControlTechnology2023, xingRealtimeOptimalScheduling2023}. The reviewed literature favoured the inclusion of \ac{ESS} storage setpoint in the action space, along with its respective output value in the observation space, with some recent works applying a specialized focus on this element, such as \cite{liuDistributedEconomicDispatch2018} which also accounts for battery health and its degradation.

  \begin{table}[H]
	\centering
	\rowcolors{2}{white}{gray!15}
	\caption{Action space of \ac{DED} systems.}
	\begin{tabular}{ccccccc}
		\toprule
		\multirow{2}{*}{\textbf{Reference}} & \multirow{2}{*}{\textbf{Model}} & \multicolumn{5}{c}{\textbf{Action Space}} \\ 
		\cmidrule(lr){3-7}
 		&  &  \textbf{Dispatch} & \textbf{Curtail} & \textbf{ESS} & \textbf{Voltage} & \textbf{Others} \\ 
		\midrule
		\cite{zhaoGraphbasedDeepReinforcement2022} & SAGE-PPO & x & x & x & & \\
		\cite{liNovelGraphReinforcement2022, chenScalableGraphReinforcement2023} & GCN-SAC & x & x & x & & \\
		\cite{hanAutonomousControlTechnology2023} & LM-SAC \& IL-SAC & x & & & x & \\
		\cite{liuDistributedEconomicDispatch2018} &  Cooperative Multi-Agent & x & & x & & \\
		\cite{leiDynamicEnergyDispatch2021} & FH-DDPG and FH-RDPG & x & & & & \\
		\cite{yangDynamicEnergyDispatch2021} & Improved DDPG & & & x & & (CHP + GB) \\
		\cite{xingRealtimeOptimalScheduling2023} & GAT-DDPG & & x & x & (SVC) & (FL) \\
		% Add more rows as needed
		\bottomrule
	\end{tabular}
	\label{tab:ded-action}
\end{table}

With respects to the observation space, the choices were sparser. The literature mainly encompasses the active power of both loads and generators, but there was a division in works that included reactive power and those who did not. Beyond that, \cite{hanAutonomousControlTechnology2023} is the only paper that considered voltage of generators and loads, which is sensible given that it also consider voltage control actions. \cite{zhaoGraphbasedDeepReinforcement2022} also considers voltage levels but only for buses and not for the individual elements, including their voltage angle as well. \par
Lastly, \cite{hanAutonomousControlTechnology2023} opted to include immediate forecasts of active power demand of loads and renewable generator output before curtailment actions. In turn, \cite{liuDistributedEconomicDispatch2018, yangDynamicEnergyDispatch2021} used time-varying electricity prices from utility grids, with the former also accounting for gas prices. \par



\begin{sidewaystable}[H]
	\centering
	\rowcolors{2}{white}{gray!15}
	\caption{Observation space of \ac{DED} systems.}
	\begin{tabular}{cccccccccp{3cm}}
		\toprule
		\multirow{2}{*}{\textbf{Reference}} & \multirow{2}{*}{\textbf{Model}} & \multicolumn{8}{c}{\textbf{Observation Space}} \\ 
		\cmidrule(lr){3-10}
		&  &  \textbf{Generators} & \textbf{Loads} & \textbf{ESS} & \textbf{Nodes} & \textbf{Lines} & \textbf{Topology} & \textbf{Timestep} & \textbf{Others}\\ 
		\midrule
		\cite{zhaoGraphbasedDeepReinforcement2022} & SAGE-PPO & $P$ and $Q$  & $P$ and $Q$ & & $P$, $Q$, $v$, and $\theta$ & $P$ and $F$ & x & & \\
		\cite{liNovelGraphReinforcement2022, chenScalableGraphReinforcement2023} & GCN-SAC & $P$ & $P$ and $Q$ & $\text{SOC}$ & & & x & x & \\
		\cite{hanAutonomousControlTechnology2023} & LM-SAC \& IL-SAC & $P$, $Q$, and $v$ & $P$, $Q$, and $v$ & & & $\text{rho}$ and $F$ & & & Forecast of $P^\text{LOAD}$ and $\overline{P^\text{RES}}$\\
		\cite{liuDistributedEconomicDispatch2018} &  Cooperative Multi-Agent & $P$ & $P$ & $\text{SOC}$ and $\text{SOH}$ & & & & & \\
		\cite{leiDynamicEnergyDispatch2021} & FH-DDPG and FH-RDPG & $P^\text{RES}$ & $P$ & $\text{SOC}$ & & & & & \\
		\cite{yangDynamicEnergyDispatch2021} & Improved DDPG & & $P$ & $\text{SOC}$ & & & & &  Price of purchasing energy and gas \\
		\cite{xingRealtimeOptimalScheduling2023} & GAT-DDPG & $P$ and $Q$ & $P$ and $Q$ & & & & x &  & \\
		% Add more rows as needed
		\bottomrule
	\end{tabular}
	\label{tab:ded-obs}
\end{sidewaystable}

\section{Conclusions} \label{sec:review-conclusion}

In this chapter, we reviewed relevant literature for this dissertation's main research topic, \ac{GRL} algorithms. \ac{GRL} is very promising field, where several different applications and techniques were already studied. \acp{GNN} architectures such as \ac{GCN} have been extensively applied with DRL algorithms for enabling feature extracting from graph-based state representations \cite{chenScalableGraphReinforcement2023, chenAutonomousExplorationUncertainty2020}. Architectures such as the GraphSAGE and other attention-based have also been successfully applied with very promising results \cite{peiEmergencyControlStrategy2023, xingRealtimeOptimalScheduling2023} in comparison with \acp{GCN}. However, less research regarding their integration with \ac{DRL} algorithms was discovered. This suggests that a possible improvement and research direction in the development of \ac{GRL} techniques might be connected with exploring the use of different \acp{GNN} architectures and using the rising attention-based techniques.

\begin{itemize}
	\item RQ1 - \textbf{How can \ac{RL} algorithms be adapted to effectively solve sequential decision-making problems on graph-based environments?} 
	\begin{itemize}
		\item RQ1.1 - \textbf{What are the existent \ac{GRL} approaches?}
		 Existent approaches ubiquitously use \acp{DRL} with \acp{GNN} for efficiently extract graph features. They can be divided into plain \ac{GCN} approaches, Attention-based approaches and others. \acp{GCN} compromise a popular method while attention-based approaches showed to be less researched but more promising \cite{xingRealtimeOptimalScheduling2023, zhaoLearningSequentialDistribution2022}. Tables \ref{tab:gcn-lit}, \ref{tab:gat-lit} and \ref{tab:other-lit} depict the reviewed \ac{GCN}, attention-based and other implementations, respectively. \\
		\item RQ1.2 - \textbf{What are their limitations?}
		 Research already depicts \ac{GRL} techniques as better performing in relation to plain \ac{RL} algorithms in graph-based environments, which portrays \ac{GRL} is a potential solution to this problem \cite{chenScalableGraphReinforcement2023, liNovelGraphReinforcement2022, zhaoGraphbasedDeepReinforcement2022, xingBilevelGraphReinforcement2023, xingGraphReinforcementLearningBased2023, xuSimulationConstraintGraphReinforcement2020}. The reviewed literature proved to be quite scarce and sparse, probably due to the novelty of the field. After a thorough review, we failed to find works that listed specific limitations of \ac{GRL} algorithms. This suggests the existence of a research gap for works with a thorough and well-documented scientific process as well as comparative and systematic studies between the different approaches, highlighting models performance in large scenarios and under topology variation. Furthermore, some papers suggested the directly embedding of the \ac{GNN} into the \ac{RL} framework \cite{xingRealtimeOptimalScheduling2023, xingBilevelGraphReinforcement2023}. \\
	\end{itemize}
	\item RQ2 - \textbf{How can \ac{GRL} algorithms improve smart grid services?}
	\begin{itemize}
		\item RQ2.1 - \textbf{How can \ac{GRL} algorithms improve \acf{DED} in power distribution grids?}  \ac{RL} algorithms are already regarded as a well established potential solution for solving economic dispatch problems \cite{pereraApplicationsReinforcementLearning2021}. We were able to find evidence of \ac{GRL} being successfully implemented in \ac{DED} systems such as \cite{chenScalableGraphReinforcement2023} and \cite{liNovelGraphReinforcement2022} in a power grid simulation context, showing efficient performance and scalability in comparison with plain \ac{DRL} models in the same issue. However, as it was the case with \ac{GRL} algorithms, the sparseness of different considerations, constraints and formalizations of the \ac{DED} problem, highlighted in section \ref{sec:review-ded} bring added complexity when comparing different approaches, since models are build with different \ac{DED} requirements. 
	\end{itemize}
\end{itemize}




