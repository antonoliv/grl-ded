\chapter*{Resumo}
%\addcontentsline{toc}{chapter}{Resumo}

Os grafos são representações que descrevem problemas e os seus objectos em domínios orientados para as redes, como as redes eléctricas, os transportes ou as redes sociais. Estas representações podem captar não só os conceitos e as suas respectivas propriedades, mas também as relações entre esses conceitos, resultando em estruturas de dados frequentemente complexas e esparsas que são especialmente úteis para representar problemas em que a topologia de uma rede desempenha um papel importante. No contexto da utilização destas estruturas em algoritmos de aprendizagem computacional, o seu desempenho torna-se dependente não só da sua conceção e seleção dos atributos relevantes e parâmetros, mas também das representações subjacentes utilizadas para captar a essência das estruturas em grafo. \par
A Aprendizagem por Reforço em Grafos ou \textit{Graph Reinforcement Learning} (GRL) é um tópico que tem merecido grande atenção por parte dos académicos nos últimos anos. Ao permitir que as técnicas de Aprendizagem por Reforço aprendam e optimizem processos de decisão sequenciais em ambientes baseados em grafos, os sistemas podem ser melhorados de forma a tirar partido das características da topologia dos grafos em domínios de aplicação associados a redes. Com os avanços no final da década de 2010 em Redes Neuronais de Grafos, ou \textit{Graph Neural Networks}, na aprendizagem e extração de representações eficientes de grafos, foram propostos métodos mais sofisticados de GRL e o tópico começou a atrair mais curiosidade dos académicos. Embora, nos últimos anos, muito trabalho tenha sido feito nesta área, a pesquisa à volta do GRL ainda é considerada estar em fase inicial. \par
Além disso, considerando os actuais desafios globais associados à sustentabilidade e aos sistemas energéticos, há uma necessidade crescente de avanços em sistemas inteligentes focados em modernizar as redes de distribuição e transmissão de energia. Atualmente, as fontes de energia renováveis desempenham um papel importante na redução da dependência dos combustíveis fósseis, o que altera a topologia dos sistemas de distribuição de energia à medida que os consumidores adquirem a capacidade de gerar energia renovável. Com as melhorias na Inteligência Artificial e Aprendizagem Computacional, os sistemas podem ser adaptados à descentralização da produção e gerir eficientemente a monitorização, distribuição e transmissão da energia. Neste trabalho, a ênfase principal reside na melhoria dos algoritmos de GRL que serão aplicados no contexto  do problema da distribuição dinâmica e económica de energia como seu principal domínio de aplicação, considerando fontes de energia renováveis e sistemas de armazenamento de energia.  \par
Desta forma, esta dissertação tem como objetivo fazer avançar a investigação existente sobre técnicas de Aprendizagem por Reforço em Grafos através de: (1) realizar uma revisão exaustiva da literatura recente relativa às várias abordagens de GRL propostas e de sistemas de distribuição dinâmica e económica de energia, de modo a obter uma perspetiva global das técnicas recentes mais avançadas e das suas limitações; (2) realizar um estudo empírico comparativo e sistemático das diferentes técnicas de GRL no problema de distribuição dinâmica e económica, considerando cenários de estudo de caso de diferentes dimensões modelados por uma simulação de uma rede de distribuição de energia eléctrica; (3) propor um modelo que melhor integre as capacidades das técnicas de \textit{Deep Reinforcement Learning} e \textit{Graph Neural Networks}, com base nos resultados do estudo empírico e com melhorias no desempenho e escalabilidade face aos modelos propostos pela literatura. \\ \\



\chapter*{Abstract}
%\addcontentsline{toc}{chapter}{Abstract}


Graphs are structures that depict problems and their objects in network-oriented domains such as power grids, transport or social networks. These representations can capture no only the concepts and their respective properties but the intricate relationships between those concepts, resulting in often complex and sparse data structures that are especially useful for representing problems where network topology plays a major role. In the context of using these structures in machine learning algorithms, their performance becomes not only dependent on its design and the selection of relevant features and parameters, but also on the underlying representations used to capture the essence of graph structures. \par
Graph Reinforcement Learning (GRL) is a topic that has earned significant attention from academics in the last few years. By enabling Reinforcement Learning techniques to learn and optimize sequential decision-making processes in graph-based environments, systems can be improved and gain the ability to leverage graph topology features in network-oriented application domains. With the advancements in the late 2010s on Graph Neural Networks on learning how to extract efficient graph representations from a given scenario, more sophisticated methods of GRL were proposed and the topic started to attract the curiosity of scholars. Although, in recent years, a lot of work has been done in this area, research on techniques is still considered to be in an early stage. \par
Furthermore, considering the current global challenges associated with sustainability and energy systems, there is an increasing need for advancements in energy-focused intelligent systems to modernize the current power grids. In the present, renewable energy sources play a major role in reducing the reliance on fossil fuels, which changes the topology of energy distribution systems as consumers gain the capability to generate renewable power. With the improvements in Artificial Intelligence and Machine Learning, systems can be adapted to the decentralization of energy production and efficiently manage the monitoring, distribution and transmission of energy systems. In this work, the primary emphasis lies on improving GRL algorithms which will be applied to solve the dynamic economic power dispatch problem as its main application domain, considering renewable energy sources and energy storage systems.  \par
In this manner, this dissertation aims to advance the existing research on Graph Reinforcement Learning techniques by: (1) conducting a thorough review of the recent literature regarding various proposed GRL approaches and Dynamic Economic Dispatch Systems to gain an overall perspective of the recent state-of-the-art techniques and their limitations; (2) performing a comparative and systematic empirical study on the different GRL techniques on the Dynamic Economic Power Dispatch problem, considering case study scenarios of different sizes modelled by a power distribution grid simulation (3)  propose a model that better integrates the capabilities of Deep Reinforcement Learning Agents with Graph Neural Networks, based on the results of the empirical study, with improvements in performance and scalability. \\ \\

\titles{Keywords}{Graph Reinforcement Learning, Graph Neural Networks, Deep Reinforcement Learning, Smart Grid, Dynamic Economic Dispatch}
\titles{ACM Classification}{Computing Methodologies $\to$ Machine Learning $\to$ Learning Paradigms $\to$ Reinforcement Learning}
