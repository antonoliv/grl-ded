

\chapter*{Resumo}
%\addcontentsline{toc}{chapter}{Resumo}

Este documento ilustra o formato a usar em dissertações na \Feup.
São dados exemplos de margens, cabeçalhos, títulos, paginação, estilos
de índices, etc. 
São ainda dados exemplos de formatação de citações, figuras e tabelas,
equações, referências cruzadas, lista de referências e índices.
Este documento não pretende exemplificar conteúdos a usar. 
É usado o \emph{Loren Ipsum} para preencher a dissertação.

Lorem ipsum dolor sit amet, consectetuer adipiscing elit. Etiam vitae
quam sed mauris auctor porttitor. Mauris porta sem vitae arcu sagittis
facilisis. Proin sodales risus sit amet arcu. Quisque eu pede eu elit
pulvinar porttitor. Maecenas dignissim tincidunt dui. Pellentesque
habitant morbi tristique senectus et netus et malesuada fames ac
turpis egestas. Donec non augue sit amet nulla gravida
rutrum. Vestibulum ante ipsum primis in faucibus orci luctus et
ultrices posuere cubilia Curae; Nunc at nunc. Etiam egestas. 

Donec malesuada pede eget nunc. Fusce porttitor felis eget mi mattis
vestibulum. Pellentesque faucibus. Cras adipiscing dolor quis
mi. Quisque sagittis, justo sed dapibus pharetra, lectus velit
tincidunt eros, ac fermentum nulla velit vel sapien. Vestibulum sem
mauris, hendrerit non, feugiat ac, varius ornare, lectus. Praesent
urna tellus, euismod in, hendrerit sit amet, pretium vitae,
nisi. Proin nisl sem, ultrices eget, faucibus a, feugiat non,
purus. Etiam mi tortor, convallis quis, pharetra ut, consectetuer eu,
orci. Vivamus aliquet. Aenean mollis fringilla erat. Vivamus mollis,
purus at pellentesque faucibus, sapien lorem eleifend quam, mollis
luctus mi purus in dui. Maecenas volutpat mauris eu lectus. Morbi vel
risus et dolor bibendum malesuada. Donec feugiat tristique erat. Nam
porta auctor mi. Nulla purus. Nam aliquam. 


\chapter*{Abstract}
%\addcontentsline{toc}{chapter}{Abstract}

Graph Reinforcement Learning is a topic that has earned significant attention and research in the last few years. Combining Deep Reinforcement Learning techniques with an underlying graph-based representation of data enables intelligent agents to learn, optimize and take actions in various scenarios where the data is network-oriented. Although a lot of work has been done on the topic in more recent years, research is still considered to be in an early stage. \par
Considering the current global challenges associated with sustainability and energy systems, there is an increasing need for advancements in energy-focused intelligent systems to modernize power distribution grids. In the present, renewable energy sources play a major role in reducing the reliance on fossil fuels, which changes the topology of energy distribution systems as consumers gain the capability to generate renewable power. Furthermore, with the improvements in Artificial Intelligence and Machine Learning, systems can be designed to adapt to the decentralization of energy production and efficiently manage energy monitoring and distribution. This translates into the transition to the \textit{smart grid}. Graph Reinforcement Learning can improve these systems, enabling the development of decision-making agents that understand the energy distribution network topology and use it to learn optimal policies for its adjacent control problems. \par
This dissertation aims to advance the existing research on Graph Reinforcement Learning techniques by proposing a model with concrete improvements to the current ones by better integrating the capabilities of Deep Reinforcement Learning Agents with Graph Neural Networks, which encompass the state-of-the-art techniques regarding Machine Learning on Graphs. \par
In this context, theoretical and empirical research on these topics is performed, as well as a thorough review of the recent literature that studies and proposes Graph Reinforcement Learning models,  gaining an overall perspective of the recent state-of-the-art techniques. Ultimately, this culminates in a proposed implementation of a deep reinforcement learning agent that uses and improves a graph neural network layer. \par
Lastly, the model is applied to a case study scenario inside the context of smart grid services to evaluate its capabilities and performance. It's expected to design a novel model that intelligently combines Graph Neural Networks with Reinforcement Learning and shows similar or better performance than the other models proposed until the present.
\\

\titles{Keywords}{Graph Reinforcement Learning, Graph Neural Networks, Deep Reinforcement Learning, Smart Grid}
\titles{ACM Classification}{Computing Methodologies $\to$ Machine Learning $\to$ Learning Paradigms $\to$ Reinforcement Learning}
