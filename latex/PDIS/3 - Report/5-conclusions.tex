\chapter{Conclusions} \label{chap:conclusions}

\section{Expected Contributions}

\begin{description}
	\item[Scientific] A systematic and comparative study of different GRL approaches
	\item[Technological] A model resulting from this study with concrete improvements over the current GRL techniques proposed so far
	\item[Application] An efficient solution for the DED problem with GRL algorithms
\end{description}

\section{SWOT Anaysis}

\begin{table}[h!]
	\centering
	\caption{SWOT Analysis}
	\begin{tabular}{|P{7cm}|P{7cm}|}
		\hline
		\textbf{Strengths} & \textbf{Opportunities} \\
		\hline
		Good understanding of the base knowledge inspiring this work & Deliver scientific contributions to the area of \ac{GRL} \\
		Well-defined evaluative dimensions & Propose a state-of-the-art \ac{GRL} model for \ac{DED} \\
		At least a third of the dissertation write-up is concluded & Identify defiencies in current \ac{GRL} approaches \\
		Well-maintained power grid simulation framework & Compare different \ac{GRL} stacks \\
		\hline
		\textbf{Weaknesses} & \textbf{Threats} \\
		\hline
		Sparse and short supply of literature & Significant Training Times for training models in large scenarios \\
		Somewhat unclear picture of the concrete architectural improvements to be made in \ac{GRL} algorithms  & \\
		\hline
	\end{tabular}
\end{table}

\section{SMART Analysis}

