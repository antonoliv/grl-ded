\chapter{Problem Statement} \label{chap:problem-statement}
In the preceeding chapter, the literature regarding \ac {GRL} was reviewed, uncovering the recently used approaches in the matter at hand and the promising algorithms and technologies in the field.  This chapter aims to drive deeper into the problems related to the topic of this dissertation. 

\section{Graph Reinforcement Learning}

\subsection{Problem}

As previously stated troughout most chapters, the main topic of study in this work is \acf{GRL} algorithms. \ac{GRL}  can be defined as a merge between concepts of Reinforcement Learning and Graph Theory. In this context, the formal definition of \ac{GRL} problem extends the \ac{MDP} with the added constraint of the environment being sensored as a graph3. Going deeper into this matter, this raises the additional question of how to efficiently represent the received observations by taking computational efficiency as well as data completeness into account, an issue studied by Graph Representation Learning. 
In this context, we consider that the observable state of the environment can be represented as a graph $G = (V, E)$, where $V$ is the set of nodes and $E \in V \times V$ the set of edges of the environment graph. Each node $v \in V$ At a given time step $t$, the agent receives an state $s_t = G_t(V_t, E_t)$ of the graph in that instant and computes the optimal action $a_t$ that maximizes the expected return $Return_t$, as defined by equation \ref{eq:expected-discounted-next}.




\section{Dynamic Economic Dispatch}

In the context of the application domain of this dissertation, which is Smart Grid Services, the methods to be studied as solutions of the \ac{GRL} problem will be applied to the Dynamic Economic Dispatch Problem. We consider this problem under high penetration of distributed generations, \textit{i.e.} with a significant number of distributed generators including renewable energy sources, namely photovoltaic and wind energy generation, and \acp{ESS}. 

\begin{equation} \label{eq:ded-problem}
	\text{min} F = \sum^T_{t=1}{F_G(t) + F_{RES}(t) + F_{ESS}(t)}
\end{equation} 


\begin{comment}
	
	* Define Constratins:
	* Power Balance
	* Conventional Generation
	* ESS Constraint
	
	* Real Power Balance Constraint
	* Real Power Operating Limits
	* Generation Unit Ramp rate Limits
	
	* Branch Flow Constraints
	* Voltage Constraints
	* RES Power Output Constraints
	
	* Define Graph
	Other considerations?
	
\end{comment}